\documentclass[12pt,letterpaper]{article}
\usepackage[spanish]{babel}
\usepackage[utf8]{inputenc}
\usepackage{enumerate}

\usepackage{vmargin}
\setpapersize{A4}
\setmargins{2.5cm}       % margen izquierdo
{0cm}                        % margen superior
{16.5cm}                      % anchura del texto
{23.42cm}                    % altura del texto
{10pt}                           % altura de los encabezados
{1cm}                           % espacio entre el texto y los encabezados
{0pt}                             % altura del pie de página
{2cm}                           % espacio entre el texto y el pie de página

\title{Tarea 01}
\author{Derek Almanza Infante}

\begin{document}
        
    \begin{center}

    {\LARGE Tarea 02 \par}
    {\Large Filtros de imagen \par}

    \end{center}

    \begin{enumerate}
        \item \textbf{Definición del problema}\\
            Un artista nos ha contratado para exponer sus imágenes únicas y mostrar por medio de distintos filtros
            de colores las imágenes ocultas en sus obras.
        \item \textbf{Análisis del problema}\\
            Analizando el problema lo que se requerirá será una interfaz gráfica, la cual contenga un botón para 
            explorar los archivos y un botón para cada filtro (rojo, azul o verde). Cada botón para que 
            funcione tendrá que jugar con los parámetros RGBA a su manera dependiendo del filtro que se desea.
        \item \textbf{Selección de la mejor alternativa}\\
            Hay varias alternativas para este reto, ya que el lenguaje que se necesita es uno que se centre en 
            interfaces gráficas. Entonces después de conocer este criterio optamos por usar Vala, un lenguaje usado
            y hecho para interfaces gráficas.
        \item \textbf{Diagrama de flujo}\\
        \item \textbf{¿Qué hizo cada quién?}\\
            -Derek: Habilitó el botón open para abrir la imagen, mejoró el diseño del código, creó las ramas de git, 
                realizó la documentación del código y realizó el PDF Latex.
            
            -Eliseo: Creó la ventana principal de la interfaz, creó el repositorio de Gitub, implementó el botón menú,
                mejoró el diseño del botón para abrir la imagen y creó el scroll del programa.
        \item \textbf{Mantenimiento}\\
            El programa en un futuro podría mejorar en complejidad y diseño tanto del código como de la interfaz, 
            además de que podríamos habilitar otras herramientas como venta en línea para las obras del cliente, ésta mejora se
            vendería como una actualización. El cobro calculado por la creación del programa sería de 4,000 pesos y por la
            mejora mencionada se vendería medio año después de la creación y sería por el costo de 4,000 pesos. Podría llegar a
            haber un mantenimiento anual el cual el primer año sería gratuito y se cobraría por otros dos años, con un valor de 
            1,000 pesos por año.
    \end{enumerate}

    La URL de la documentación de Vala es:\\
    - https://valadoc.org/

\end{document}
